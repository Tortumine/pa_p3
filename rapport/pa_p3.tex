\documentclass[10pt]{article}
\usepackage[T1]{fontenc}
\usepackage[francais]{babel}
\usepackage{array}
\usepackage{shortvrb}
\usepackage{listings}
\usepackage[fleqn]{amsmath}
\usepackage{amsfonts}
\usepackage{fullpage}
\usepackage{enumerate}
\usepackage{graphicx}             % import, scale, and rotate graphics
\usepackage{subfigure}            % group figures
\usepackage{alltt}
\usepackage{url}
\usepackage{indentfirst}
\usepackage{eurosym}
\usepackage{amsmath} 
\usepackage{float}
\usepackage{caption}

%Définition du c pour les "listings"
\usepackage{listings}
\usepackage{xcolor}
\definecolor{mGreen}{rgb}{0,0.6,0}
\definecolor{mGray}{rgb}{0.5,0.5,0.5}
\definecolor{mPurple}{rgb}{0.58,0,0.82}
\definecolor{backgroundColour}{rgb}{0.95,0.95,0.92}

\lstdefinestyle{CStyle}{
    backgroundcolor=\color{backgroundColour},   
    commentstyle=\color{mGreen},
    keywordstyle=\color{magenta},
    numberstyle=\tiny\color{mGray},
    stringstyle=\color{mPurple},
    basicstyle=\footnotesize,
    breakatwhitespace=false,         
    breaklines=true,                 
    captionpos=b,                    
    keepspaces=true,                 
    numbers=left,                    
    numbersep=5pt,                  
    showspaces=false,                
    showstringspaces=false,
    showtabs=false,                  
    tabsize=2,
    language=C
}


%\usepackage[french,onelanguage,ruled,vlined]{algorithm2e}
%\usepackage{clrscode3e}
%\usepackage{algorithm, algpseudocode}
%\usepackage{tabular}
\usepackage[utf8]{inputenc}
\usepackage{clrscode3e}
%\usepackage{algpseudocode}

% changement de la numerotation
\setcounter{secnumdepth}{5}
%\renewcommand{\thechapter}{\Alph{chapter}}
%\renewcommand{\thesection}{\Roman{section})}
%\renewcommand{\thesubsection}{\arabic{subsection})}


%\title{\textbf{INFO2050 : Rapport numéro un programmation avancée}}
%\author{Antoine Sadzot}
%\date{30-10-17}
\begin{document}
\begin{titlepage}

   \begin{figure}[htbp]
      \centering
      \includegraphics{uliege-logo-couleurs-300.jpg}
   \end{figure}
  	
  	\hfill

	\begin{center}
		\vfill
		\textbf{
		\Huge{INFO2050-1 - Programmation Avancée}}\\
		\bigskip
		\huge{Projet 3: Mise en page automatique d'une bande dessinée}\\
		\bigskip %saut de ligne
		\smallskip
		\Large{Aliaksei Mazurchyk\\Antoine Sadzot}\\
		\bigskip
		\smallskip
		\large{\today}\\%date
		\vfill
		\large{Université de Liège}
	\end{center}
\end{titlepage}
\clearpage
\clearpage

\section{Algorithme par programmation dynamique}
\subsection{Approche par recherche exhaustive}
\subsubsection{Comic}
\subsubsection{Seam carving}
Pour créer une couture, on part d'un pixel à la première rangée. De là, on a le choix entre trois directions pour continuer la couture à la rangée suivante. Pour avancer d'encore une rangée, on a encore le choix entre trois directions. On a donc (en ne prenant pas en compte les bords gauche et droit) une complexité $m*3^n$ où m est le nombre de pixels de départ à la première rangée et n le nombre de pixels en hauteur.

\subsection{Formulation récursive de la fonction de coût}
\subsubsection{Comic}
\subsubsection{Seam carving}
$$
C(i,j) = \left\{
	\begin{array}{ll}
		E(0,j) & si\ i==0 \\
		E(i,j)+min(C(i-1,j-1),C(i-1,j),C(i-1,j+1)) & sinon
	\end{array}
\right.
$$

\subsection{Graphe des appels récursifs pour un problème de petite taille}
\subsubsection{Comic}
\subsubsection{Seam carving}
Graphe des appels récursif
 \begin{figure} [h]
      \centering
      \includegraphics[scale=0.5]{GrapheSeamCarving.png}
   \end{figure}

\subsection{Algorithme des fonctions de coût}
\subsubsection{Comic}
\subsubsection{Seam carving}

\subsection{Complexité en temps et en espace}
\subsubsection{Comic}
\subsubsection{Seam carving}

\section{Fonctions de réduction et d'augmentation d'images}
\subsection{Choix d'implémentation}
\subsection{Complexité en temps et en espace}
\end{document}
